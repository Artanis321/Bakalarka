The bachelor thesis deals with current techniques for hiding the activity of malware in compromised systems. In the introduction of the thesis we describe the malware and individual techniques for hiding, which are most often used in recent years. In this part of the thesis, we dealt mainly with samples that were detected during the last year. The main objective of the bachelor thesis was to design a detection algorithm of a selected technique. The method that we decided to examine in more detail is the so-called \ textit {process hollowing}. The proposed algorithm is based on the principle of monitoring selected Windows API calls, which can be used to hide the malware itself in a normal process. The basis of the algorithm is a finite state machine, which evaluates the presence of the \ textit {process hollowing} technique by monitoring the sequence of typical API calls used to implement the method. In the next part we describe the algorithm itself, the individual modules of the detection application and the way in which we implemented them. In the final part of thesis, we present the results we achieved in experiments with real samples of malware.
